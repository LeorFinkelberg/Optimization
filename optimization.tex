\documentclass[%
	11pt,
	a4paper,
	utf8,
	%twocolumn
		]{article}	

\usepackage{style_packages/podvoyskiy_article_extended}


\begin{document}
\title{Специальные вопросы оптимизации}

\author{\itshape Подвойский А.О.}

\date{}
\maketitle

\thispagestyle{fancy}

Здесь приводятся заметки по специальным вопросам теории оптимизации


\shorttableofcontents{Краткое содержание}{1}

\tableofcontents






\listoffigures\addcontentsline{toc}{section}{Список иллюстраций}

% Источники в "Газовой промышленности" нумеруются по мере упоминания 
\begin{thebibliography}{99}\addcontentsline{toc}{section}{Список литературы}
	\bibitem{lutz:learningpython-2011}{\emph{Лутц М.} Изучаем Python, 4-е издание. -- Пер. с англ. -- СПб.: Символ-Плюс, 2011. -- 1280~с. }
	
	\bibitem{burkov:2020}{\emph{Бурков А.} Машинное обучение без лишних слов. -- СПб.: Питер, 2020. -- 192 с.}
		
	\bibitem{beazley:python-2010}{\emph{Бизли Д.} Python. Подробный справочник. -- Пер. с англ. -- СПб.: Символ-Плюс, 2010. -- 864~с. }
\end{thebibliography}

\end{document}
